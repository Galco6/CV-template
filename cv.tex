%!TEX TS-program = xelatex

%%%%%%%%%%%%%%%%%%%%%%%%%%%%%%%%%%%%%%%%%%%%%%%%
% CV template
% Originally created by Adrien Friggeri
% Improved by Carmine Benedetto
% Improved by Víctor Galvín 
%%%%%%%%%%%%%%%%%%%%%%%%%%%%%%%%%%%%%%%%%%%%%%%%

\documentclass[]{cv-class}
\usepackage{afterpage}
\usepackage{fontawesome}
\usepackage{tcolorbox}
\usepackage{graphicx}
\usepackage{hyperref}
\usepackage{color}
\usepackage{xcolor}
\hypersetup{
    colorlinks=true,
    linkcolor=blue
}
\addbibresource{bibliography.bib}
\RequirePackage{xcolor}
\definecolor{pblue}{HTML}{0395DE}

\begin{document}
\header{Víctor}{ Galvín Coronil}
      {Ingeniero Electrónico Industrial}

% Fake text to add separator
\vspace{1.15cm}
\fcolorbox{white}{gray}{\parbox{\dimexpr\textwidth-2\fboxsep-2\fboxrule}{%
.....
}}

% In the aside, each new line forces a line break
\begin{aside}
  \includegraphics[scale=0.20]{img/CV.png}}
    ~
  \vspace{0.65cm}

  \section{Teléfono}
   \faMobile\hspace{0.4cm}676 331 826
    ~
  \section{Correo}
   \faSendO\hspace{0.2cm}\href{mailto:victor.galvinco@gmail.com}{victor.galvinco@gmail.com}
    ~
  \section{Enlaces}
	\vspace{0.10cm}
    \faLinkedinSquare\hspace{0.2cm}\href{https://linkedin.com/in/victorgalco}{linkedin.com/in/victorgalco}
	\vspace{0.10cm}
    \faGithub\hspace{0.2cm}\href{https://github.com/Galco6}{github.com/Galco6}
    ~
    \section{Idiomas}
    \textbf{Español} - Nativo\\
    \textbf{Inglés} - Intermedio (Certificado B1)\\
    \textbf{Francés} - Básico\\
    ~
    \section{Aptitudes}
    \textbf{Electrónica}\\
    \framebox{Microcontroladores}
    \framebox{ARM}
    \framebox{AVR}
    \framebox{Diseño electrónico}
    \framebox{FPGA}
    \framebox{IoT}\\
    \vspace{0.20cm}
    \textbf{Programación}\\
    \framebox{C/C++}
    \framebox{Ensamblador}
    \framebox{VHDL}\\
    \vspace{0.20cm}
    \textbf{Software}\\
    \framebox{Eagle}
    \framebox{LabView}
    \framebox{PSpice}
    \framebox{Simulink}
    \framebox{Matlab}\\
    \vspace{0.20cm}
    \textbf{Ofimática}\\
    \framebox{\LaTeX}
    \framebox{Excel}
    \framebox{Word}
    
    \section{Otros}
    Permiso de conducción B

\end{aside}

\vspace{0.75cm}
\section{Experiencia}
\begin{entrylist}
  \entry
    {Jul. 18 - Now}
    {Becario - Ingeniero Diseño Electrónico}
    {Universidad de Cádiz}
    {Diseño, desarrollo e implementación de dispositivos electrónicos en la línea de \emph{"Sensores y sistermas de tele-medición en tiempo real"}\\\\
    Actividades realizadas:\\
    \begin{itemize}
    \item Diseño de placas de circuito impreso dotados de microcontroladores, sensores,
    transceptores de comunicación y etapas de regulación y carga de potencia.
    \item Montaje de PCB haciendo uso de las diferentes técnicas y equipos para la soldadura de componentes SMD, entre ellas, estaciones de soldadura de aire caliente y horno de refusión por infrarrojos.
    \item Programación de microcontroladores con aquitectura ARM.
    \item Redacción de documentación técnica diversa.\\\\
    \end{itemize}
    }
  \entry
    {Oct. 16 - Jun. 18}
    {Alumno Colaborador}
    {Universidad de Cádiz}
    {Participación en proyectos de investigación docente propios del grupo de investigación \emph{Diseño de Circuitos Microelectrónicos} a través del programa de Alumno Colaborador por la Universidad de Cádiz\\
    }
\end{entrylist}

\section{Formación}
\begin{entrylist}
  \entry
    {2014 - 2019}
    {Grado en Ingeniería Electrónica Industrial}
    {Universidad de Cádiz}
    {Grado en Ingeniería Electrónica Industrial especializado en asignaturas de diseño electrónico\\
    \emph{Trabajo Fin de Grado: "Configuración, implementación y mejora de un dispositivo emulador de circuitos y sistemas en tiempo real"
    }.\\}
    
\end{entrylist}

\section{Certificaciones}
\begin{entrylist}
  \entry
    {Ag. 14}
    {Preliminary English Test (PET)}
    {Cambridge English Languaje Assessment}
    {Cambridge English Entry Level Certificate in ESOL International (Entry 3) (Preliminary)}
\end{entrylist}

\begin{flushright}
\vspace{1.20cm}
\emph{Víctor Galvín Coronil}
\end{flushright}
\begin{flushright}
\emph{Marzo, 2019}
\end{flushright}

\end{document}
